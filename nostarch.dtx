% \iffalse
% $Id$
%
% Copyright 2008-2020, Boris Veytsman
% This work may be distributed and/or modified under the
% conditions of the LaTeX Project Public License, either
% version 1.3 of this license or (at your option) any 
% later version.
% The latest version of the license is in
%    http://www.latex-project.org/lppl.txt
% and version 1.3 or later is part of all distributions of
% LaTeX version 2005/12/01 or later.
%
% This work has the LPPL maintenance status `maintained'.
%
% The Current Maintainer of this work is Boris Veytsman,
% <borisv@lk.net> 
%
% This work consists of the file nostarch.dtx and the
% derived file nostarch.cls
%
% \fi 
% \CheckSum{0}
%
%
%% \CharacterTable
%%  {Upper-case    \A\B\C\D\E\F\G\H\I\J\K\L\M\N\O\P\Q\R\S\T\U\V\W\X\Y\Z
%%   Lower-case    \a\b\c\d\e\f\g\h\i\j\k\l\m\n\o\p\q\r\s\t\u\v\w\x\y\z
%%   Digits        \0\1\2\3\4\5\6\7\8\9
%%   Exclamation   \!     Double quote  \"     Hash (number) \#
%%   Dollar        \$     Percent       \%     Ampersand     \&
%%   Acute accent  \'     Left paren    \(     Right paren   \)
%%   Asterisk      \*     Plus          \+     Comma         \,
%%   Minus         \-     Point         \.     Solidus       \/
%%   Colon         \:     Semicolon     \;     Less than     \<
%%   Equals        \=     Greater than  \>     Question mark \?
%%   Commercial at \@     Left bracket  \[     Backslash     \\
%%   Right bracket \]     Circumflex    \^     Underscore    \_
%%   Grave accent  \`     Left brace    \{     Vertical bar  \|
%%   Right brace   \}     Tilde         \~} 
%
%\iffalse
% Taken from xkeyval.dtx
%\fi
%\makeatletter
%\def\DescribeOption#1{\leavevmode\@bsphack
%              \marginpar{\raggedleft\PrintDescribeOption{#1}}%
%              \SpecialOptionIndex{#1}\@esphack\ignorespaces}
%\def\PrintDescribeOption#1{\strut\emph{option}\\\MacroFont #1\ }
%\def\SpecialOptionIndex#1{\@bsphack
%    \index{#1\actualchar{\protect\ttfamily#1}
%           (option)\encapchar usage}%
%    \index{options:\levelchar#1\actualchar{\protect\ttfamily#1}\encapchar
%           usage}\@esphack}
%\def\DescribeOptions#1{\leavevmode\@bsphack
%  \marginpar{\raggedleft\strut\emph{options}%
%  \@for\@tempa:=#1\do{%
%    \\\strut\MacroFont\@tempa\SpecialOptionIndex\@tempa
%  }}\@esphack\ignorespaces}
%\makeatother
%
% \MakeShortVerb{|}
% \GetFileInfo{nostarch.dtx}
% \newcommand{\progname}[1]{\textsf{#1}}
% \title{\LaTeX{} Style For \emph{No Starch Press}
%   \thanks{\copyright 2020, Boris Veytsman}}
% \author{Boris Veytsman\thanks{%
% \href{mailto:borisv@lk.net}{\texttt{borisv@lk.net}},
% \href{mailto:boris@varphi.com}{\texttt{boris@varphi.com}}}} 
% \date{\filedate, \fileversion}
% \maketitle
% \begin{abstract}
%   This package provides class for typesetting books for No Starch
%   Press, \url{http://www.nostarch.com}
% \end{abstract}
% \tableofcontents
%
% \clearpage
%
% \changes{v0.1}{2008/02/27}{First fully functional version} 
% \changes{v0.5}{2008/03/19}{Documentation update} 
% \changes{v0.6}{2008/03/21}{Documentation update} 
% \changes{v1.0}{2008/04/07}{First public release} 
% \changes{v1.1}{2008/05/24}{Added interface to listings package and
% upquote package.  Updated documentation} 
% \changes{v1.2}{2020/04/29}{Updated interface with listings package} 
%
%
%\section{Introduction}
%\label{sec:intro}
%
% This is the ``official'' \LaTeX{} style for \emph{No Starch Press,}
% \url{http://www.nostarch.com}.  The class is intended to typeset
% books for this publisher---both hard copies and electronic books.
%
% The suite provides main class, |nostarch.cls|, hyperref interface
% |nshyper.sty| and MakeIndex style |nostarch.ist|.
%
%\section{User's Guide}
%\label{sec:manual}
%
%
%\subsection{Installation}
%\label{sec:installation}
%
% The installation of the class follows the usual
% practice~\cite{TeXFAQ} for \LaTeX{} packages:
% \begin{enumerate}
% \item Run \progname{latex} on |nostarch.ins|.  This will produce the
% following files:
% \LaTeX{} class |nostarch.cls|, \LaTeX{} style |nshyper.sty| and
% MakeIndex style |nostarch.ist|.
% \item Put the files |nostarch.cls|, |nshypher.sty| and
% |nostarch.ist| to
% the places where \LaTeX{} and MakeIndex can find them (see
% \cite{TeXFAQ} or the documentation for your \TeX{}
% system).\label{item:install} 
% \item Update the database of file names.  Again, see \cite{TeXFAQ}
% or the documentation for your \TeX{} system for the system-specific
% details.\label{item:update}
% \item The file |nostarch.pdf| provides the documentation for the
% package (this is the file you are probably reading now).
% \end{enumerate}
% As an alternative to items~\ref{item:install} and~\ref{item:update}
% you can just put the files in the working directory where your
% |.tex| file is.
% 
% To typeset the books in the distinctive \emph{No Starch Press} style
% you will need the fonts used by them.  Unfortunately these fonts
% (New Baskerville, Futura, The Sans Mono Condensed and Dogma) are not
% free.  You may buy them from the vendor or license from the
% publisher to typeset your book.  Please contact \emph{No Starch
% Press} directly for the arrangements.
%
% The class uses some other \LaTeX{} classes or packages.  Most
% probably, some of them they are already installed on your system.
% If not (or if their versions are very old), you need to download and
% install them.  Here is the list:
% \begin{enumerate}
% \item Font support packages \progname{nbaskerv}~\cite{NbaskervFont},
% \progname{futurans}~\cite{FuturansFont},
% \progname{dogma}~\cite{DogmaFont} and
% \progname{thsmc}~\cite{ThsmcFont}.
% \item \progname{booktabs} package~\cite{Fear05:Booktabs}.
% \item \progname{caption} package~\cite{Sommerfeldt07:Caption}.
% \item \progname{fancyhdr} package~\cite{Oostrum04:Fancyhdr}.
% \item \progname{fancyvrb} package~\cite{VanZandt98:Fancyvrb}.
% \item \progname{graphics} package~\cite{Carlisle05:Graphics}.
% \item \progname{ifpdf} package~\cite{Oberdiek06:Ifpdf}.
% \item \progname{listings} package~\cite{Heinz07:Listings}.
% \item \progname{ragged2e} package~\cite{Schroeder03:Ragged2e}.
% \item \progname{upquote} package~\cite{Covington03:Upquote}.
% \end{enumerate}
% These packages are called internally by |nostarch.cls|;  you do not
% need to explicitly call them from your document.
%
%
% 
%
%
%\subsection{Invocation}
%\label{sec:invocation}
%
% To use the class, put in the preamble of your document
% \begin{flushleft}
% |\documentclass[|\meta{options}|]{nostarch}|
% \end{flushleft}
%
% \DescribeOptions{cfonts,nocfonts} 
% As discussed above, \emph{No  Starch Press} uses a number of
% commercial fonts to typeset their  books.  
% The option |cfonts| (default) tells \LaTeX{} that you do have these
% commercial fonts.  On the other hand, the option |nocfonts| instructs
% \LaTeX{} 
% to use freely available fonts for typesetting.  Of course in this
% case the result will not look like a \emph{No Starch Press} Book.
%
% 
% \DescribeOptions{8pt,9pt,10pt,11pt,12pt}
% The fontsize changing options (|8pt|, |9pt|, \dots, |12pt|) have no
% effect other than producing a warning in the log.
%
%
%
% \emph{No Starch Press} uses a special paper size.  If you process the
% manuscript with \progname{pdflatex} to produce PDF output, the paper
% dimensions will be automatically set up by the class.  However, if
% you use \progname{latex} and \progname{dvips}, you need to tell
% \progname{dvips} what paper size to choose.  One way to do this is
% to add to the \progname{dvips} options the following:
% \begin{verbatim}
%   -T 7in,9.25in
% \end{verbatim}
% 
%\subsection{Use of \progname{hyperref} Package}
%\label{sec:guide:hyperref}
%
% The books typeset by \emph{No Starch Press} can be sold either as
% traditional printed and bound hard copies, or as PDF files.  If you,
% the author, envisage the second possibility, you may want additional
% possibilites provided by \progname{hyperref}
% package~\cite{Rahtz06:Hyperref}, like ``clickable'' entries in the
% table of context and index, etc.  The provided package |nshyper.sty|
% adds the functionality of \progname{hyperref} to the class.  This
% package is highly recommended for the books that are going to be
% distributed in an electronic form.
%
% To use the package, just add the line |\uspackage{nshyper}| to your
% preamble.  This package should be loaded last since it redefines
% internals of many other packages.  Note that a direct call of
% \progname{hyperref} by |\usepackage{hyperref}| will not work since
% \progname{nshyper} patches hyperref to work with our class.
% 
%
%
%\subsection{Front Matter}
%\label{sec:guide:frontmatter}
%
% \DescribeMacro{\frontmatter}
% A book can be divided into three parts:  front matter, main matter
% and back matter.  Front matter starts with the command
% |\frontmatter|.  Normally this is the first command in the class.
% Front matter contains half page, title page, copyright page, brief
% contents, detailed contents, foreword(s) and other stuff which is
% traditionally typeset on roman-numbered pages.  The commands in main
% matter can be divides into \emph{declarations} and
% \emph{directives.}  Declarations do not typeset anything.  Rather,
% they up parameters for the typesetting.  Directives use them to
% typeset the stuff.
%
%
%\subsubsection{Front Matter Declarations}
%\label{sec:guide:decls}
%
%
%
% \DescribeMacro{\title} 
% There are five declarations defined by |nostarch.cls|: |\title|,
% |\subtitle|, 
% |\author|, |\nostarchlogo| and |\nostarchlocation|.  The macro
% |\title|, like the standard \LaTeX{} |\title|, sets the title of the
% book, for example |\title|\texttt{\{The book about everything\}}.
%
% \DescribeMacro{\subtitle}
% The macro |\subtitle| takes one argument and sets up the subtitle of
% your book, for example |\subtitle|\texttt{\{A Guide For
% Perplexed\}}. 
%
% \DescribeMacro{\author}
% The command |\author| has just one argument, which is the list of
% authors, separated by commas and (for the last author) the word
% ``and''.  Do not use the command |\and| defined in the standard
% \LaTeX.  For example,
% |\author|\texttt{\{Ralph Alpher, Hans Bethe and George Gamow\}}.
% 
% \DescribeMacro{\nostarchlogo} 
% Due to copyright reasons the logo of \emph{No Starch Press} is not
% distribuded with the package.  You should get the logo as a graphics
% file from the publishers and put in the directory where \LaTeX{} can
% find it (for example, the directory with the other sources of your
% book).  The command |\nostarchlogo| has one argument: the name of
% the file, for example, |\nostarchlogo{biglogo.pdf}|.
% The file will be automatically scaled to the proper size.
%
% \DescribeMacro{\nostarchlocation}
% \emph{No Starch Press} books are usually published in San
% Francisco.  If you need to change this setting, use
% |\nostarchlocation| declaration, for example,
% |\nostarchlocation|\texttt{\{San Francisco;  London; Tokyo\}}.
% 
%
%
%\subsubsection{Front Matter Directives}
%\label{sec:guide:directives}
%
%
%
% \DescribeMacro{\makehalftitle}
% The command |\makehalftitle| creates \emph{half title page.}
% This is normally the first page of the book.
%
% \DescribeMacro{\maketitle}
% The command |\maketitle| creates \emph{title page.}  It uses logo
% and full title of the book.
%
% \DescribeMacro{copyrightpage} 
% The copyright page is the \emph{verso} of the title page.  It has
% information about copyright, edition history, the credits of the
% people who helped you to make the book, CIP information, etc.  The
% exact contents of the copyright page differ from book to book.
% Therefore the class does not try to create this page automatically,
% and rather sets up the typographic style for arbitrary contents.
% For this purpose the environment |copyrigthtpage| is used.  So just
% put the information provided by the publisher between
% |\begin{copyrightpage}| and |\end{copyrightpage}|.  Use
% |\textbf{\sffamily title}| for the title of the book on the
% copyright line.
%
% \DescribeMacro{cipblock}
% One of important parts of the copyright page is \emph{CIP block.}
% It contains catalogue data for your book (CIP means Cataloguing in
% Publication, by the way).  Again, the contents of this field will be
% provied by the publishers;  you just need to put them inside
% |cipblock| environment, i.e. between |\begin{cipblock}| and
% |\end{cipblock}|.  To make your life easier, this block is typeset
% verbatim, so line ends and spaces are typeset as is.
%
%
% \DescribeMacro{dedicationpage}
% Dedications, if any, are included in the special dedication page,
% for which a special environment |dedicationpage| is provided.  
%
% \DescribeMacro{\brieftableofcontents}
% \DescribeMacro{\tableofcontents}
% The standard \LaTeX{} provides just one table of contents.  This
% class uses two tables:  brief contents and contents in detail.  They
% are correspondingly typeset by the commands |\brieftableofcontents|
% and |\tableofcontents|. 
% 
%
% \DescribeMacro{\sectionnumberwidth}
% \DescribeMacro{\subsectionnumberwidth}
% In most books by \emph{No Starch Press} sections and subsections are
% unnumbered.  However, sometimes the books use numbered sections or
% subsections.  In these cases you may want to adjust the widths left
% for section and subsection numbers in the detailed table of
% contents.  They are set by
% |\sectionnumberwidth| and |\subsectionnumberwidth|, and can be
% changed with the usual |\setlength| command.
%
%
% Forewords and similar front matter materials are unnumbered
% chapters: |\chapter*[Foreword by R. E.~Viewer]{Foreword}|.
% 
%
% Front matter with its formal parts like copyright page, CIP
% information, etc., can be daunting, especially for novice authors.
% Feel free to use the sample enclosed and ask the publisher for
% advice. 
%
%\subsection{Main Matter}
%\label{sec:guide:mainmatter}
%
% \DescribeMacro{\mainmatter}
% The command |\mainmatter| finishes the front matter and starts the
% main matter of the book.  The main matter is divided into chapters,
% sections, subsections, etc.  They are discussed below.
%
%
%\subsubsection{Parts, Chapters and Sections}
%\label{sec:guide:sections}
%
% \DescribeMacro{\chapter}
% \DescribeMacro{\section}
% \DescribeMacro{\subsection}
% \DescribeMacro{\subsubsection}
% \emph{No Starch Press} books have chapters, sections, subsections and
% subsubsections.  The chapters are usually numbered, but sections ande
% below are not.  The commands for them are the same as in the
% standard \LaTeX: |\chapter|\oarg{short title}\marg{long title},
% |\section|\oarg{short title}\marg{long title}, etc. For
% example:
% \begin{verbatim}
% \section[Building Lego Fire Track]{How To Build A Lego Fire Track 
%    And Why It Is Fun}
% \end{verbatim}
% The ``starred'' forms |\chapter*|, |\section*| etc. are used when
% the corresponding division has no number (like unnumbered chapters in
% the front matter).  Since our sections, subsections and
% subsusbsections are unnumbered anyway, this form has meaning only
% for chapters (but can be used for other divisions as well).
%
% \DescribeMacro{\chapterart}
% \DescribeMacro{\chapterartfile}
% \emph{No Starch Press} books use  ``circular art'' for chapter
% openings (except for unnumbered chapters in the front matter).  The
% command |\chapterart| should be issued immediately 
% after |\chapter| command and before the first paragraph.  Its
% argument is the artwork used for the current chapter:
% |\chapterart|\marg{\LaTeX{} commands}.  In the
% simplest case it is just inclusion of an external graphics file, for
% example: 
% \begin{verbatim}
% \chapterart{\includegraphics[width=1.264in]{firetrack.jpg}}
% \end{verbatim}
% Actually, this simplest case is so ubiquitous, that there is a
% special command |\chapterartfile|\marg{filename} for it, so the
% previous example can be simplified as
% \begin{verbatim}
% \chapterartfile{firetrack.jpg}
% \end{verbatim}
% However, you can use instead any \LaTeX{} commands  to produce
% the artwork.  Rememer that it must be $1.264''\times1.222''$.
%
% The first paragraph after chapter start is typeset in larger size
% font than the body font.  If this paragraph is too long, you may
% need to manually change the font size back in the middle of it.  The
% simplest way to do this is the combination
% |\par\noindent\normalfont| in a strategic place.
%
%
% \DescribeMacro{\part}
% Sometimes larger books are separated into parts.  The command
% |\part|\marg{Title} is used to typeset parts.
%
%\subsubsection{Environments}
%\label{sec:guide:lists}
%
%
% \DescribeMacro{itemize}
% \DescribeMacro{enumerate}
% \DescribeMacro{description}
% \DescribeMacro{note}
% The class offers standard |itemize|, |enumerate| and |description|
% environment.  There is an environment specific for it:  |note|.  It
% is intended for ``asides'':
% \begin{verbatim}
% \begin{note}
%   Do not forget to plug off the computer before doing any
%   modifications! 
% \end{note}
% \end{verbatim}
% The material in a |note| will be typeset in italics with bold
% ``NOTE'' on the margin.
%
%
%\subsubsection{Tables and Figures}
%\label{sec:guide:floats}
%
% There are several things to keep in mind when using tables and
% figures with the class.
%
% The tables and figures are not centered.  Neither are their
% captions. The captions for figures go \emph{below} the figures, the
% captions for tables go \emph{above} the tabular data.  
%
% If a caption for a table or figure turns out to be wider than the
% body, you might want to split the caption into lines.  Remember,
% however, that the mandatory argument to |\caption| becomes
% ``robust'' in the \LaTeX{} parlance~\cite{Lamport94} only when there
% is the optional one.  Therefore the correct way to split a caption
% is the following:
% \begin{verbatim}
% \caption[A Long Caption With Lines To Split]{%
%   A Long Caption\\ With Lines To Split}
% \end{verbatim}
% 
% Typesetting tables, unfortunately, is rarely done right, and
% standard \LaTeX{} is not an exception.  The authors are urged to
% read the introduction to \progname{booktabs}
% package~\cite{Fear05:Booktabs}.  To summarize,
% \begin{enumerate}
% \item Never ever use vertical rules.
% \item Never ever use double rules.
% \item Use only |\toprule|, |\midrule| and |\bottomrule| for tables. 
% \end{enumerate}
% \DescribeMacro{\tbfont}
% \DescribeMacro{\thfont}
% \DescribeMacro{tabular}
% \emph{No Starch Press} uses special fonts for table body and table
% header.  Since there are too many tabular-like environments
% (|tabular|, |tabbing|, |longtable|, \dots) we do not redefine them
% switch on these fonts automatically, but rather provide two font
% switching commands.  The command |\tbfont| should be used
% \emph{before} tabular environment, and the command |\thfont| should
% be used in all header cells.  Also, usually the table entries are
% flushed left, so you might want to put |@{}| before the first column
% declaration in |\begin{tabular}| to suppress the padding on the
% left. Here is an example of a properly
% done table layout:
% \begin{verbatim}
% \begin{table}
%   \caption{Starch Content of Foods}
%   \label{tab:starch}
%   \tbfont
%   \begin{tabular}{@{}lr}
%     \toprule
%      \thfont Product        & \thfont Starch Content, \% \\ 
%     \midrule
%      Bran (wheat)           & 23.0\\
%      Brown rice (raw)       & 80.0\\
%      Brown bread (average)  & 41.3\\
%      White bread (average)  & 46.7\\
%     \bottomrule
%   \end{tabular}
% \end{table}
% \end{verbatim}
% 
%
%
%\subsubsection{Code Fragments}
%\label{sec:guide:code}
%
% The class provides two facilities for including code fragments in
% the books.  
%
% \DescribeMacro{Code}
% For short unformatted code fragments the  \progname{fancyvrb}
% package~\cite{VanZandt98:Fancyvrb} facilities are recommended.  This
% package is automatically automatically loaded by the class.  Long code
% fragments should be 
% separated by rules;  the class defines verbatim-like |Code|
% environment, which does exactly this:
% \begin{verbatim}
% \begin{Code}
%   main() {
%        printf("Hello, World\n");
%   }
% \end{Code}
% \end{verbatim}
% See the documentation of~\cite{VanZandt98:Fancyvrb} about many
% features of this package.
%
% For longer code fragments the \progname{listings}
% package~\cite{Heinz07:Listings} is more appropriate.  It is also
% automatically loaded.  This package provides a number of useful
% features, well describe in its manual.  We will briefly mention
% several of them, which might be of interest for the authors of
% \emph{No Starch Press}:
% \begin{enumerate}
% \item Automatic code formatting according to the rules of the chosen
% program language (for example, |\lstset{language=C}| or
% |\lstset{language=Lisp}|).
% \item A rich set of different frames:  ruled code with
% |frame=lines|, boxed code with |frame=single| etc.
% \item An option to add captions to the listings (the class
% automatically formats the captions according to the style of
% \emph{No Starch Press}).
% \item An option to make a listing floating.
% \item An option to automatically number code lines.
% \item An option to add list of listings (with the command
% |\lstlistoflistings|). 
% \end{enumerate}
% 
% To make the code in |\verb| commands and |verbatim|-like
% environments be typeset with the \progname{listings} package, you
% may want to  set |fancyvrb=true| in the preamble of your document.
%
% It should be noted that the authors are recommended \emph{not} to
% use |\texttt| or |\tt| for inline code fragments.  The proper way to
% add inline code is |\verb| or |\lstinline| from the
% \progname{listings} package.  The difference is obvious, for
% example, in quote characters.
%
%
%\subsection{Back Matter}
%\label{sec:guide:backmatter}
%  
% \DescribeMacro{\backmatter}
%  Back matter includes bibliography, index, update page, colophon and
%  other material.  It starts with the command |\backmatter|.
%
%
%
%\subsubsection{Bibliography}
%\label{sec:guide:bibliography}
% 
% \emph{No Starch Press} at present has no special style for
% bibliography, so the authors have a relative freedom to choose one.
% However, it would be a good idea to use the \progname{natbib}
% package~\cite{Daly07:Natbib}.  This class should work with
% \progname{natbib}.
%
% No support for per chapter bibliography lists is provided by the
% package. 
%
%
%\subsubsection{Index}
%\label{sec:guide:index}
%
% The books published by \emph{No Starch Press} usually have index.
% This package provides special style file |nostarch.ist| for index
% formatting.
%
%
% \DescribeMacro{\makeindex} 
% \DescribeMacro{\printindex} 
% To create index:
% \begin{enumerate}
%
% \item Put in the preamble of your
% document the command |\makeindex|. 
% \item In the body of your document use |\index| commands as
% described in~\cite{Lamport94}.
% \item Put in the Back Matter the command |\printindex| at the place
% you want the index to be printed.
% \item After a \progname{latex} run \progname{makeindex} with the
% style |nostarch.ist|.  In a Unix system the corresponding command is
% \begin{verbatim}
% makeindex -s nostarch.ist mydocument
% \end{verbatim}
% or, in a more verbose way
% \begin{verbatim}
% makeindex -s nostarch.ist -o mydocument.ind mydocument.idx
% \end{verbatim}
% (here we assume that your main file is |mydocument.tex|).  
%
% Refer to the documentation of your \TeX{} system for the way to run
% \progname{makeindex} on your computer.
% \end{enumerate}
% 
%
%\subsubsection{Updates and Colophon}
%\label{sec:guide:updates}
%
% \DescribeMacro{\updatespage}
% \emph{No Starch Press} books usually have updates page, which lists
% the Web address for updates, errata, etc.  The command
% |\updatespage| starts such page.  You are free to choose any text
% there.  Note that if you use the package
% \progname{nshypher}~\ref{sec:guide:hyperref}, then the command
% |\url|\marg{URL} becomes a hyperlink in the electronic edition.
%
%
% \DescribeMacro{\colophon}
% The last part of Back Matter is usually a \emph{colophon} which
% describes how the book was printed and typeset.  The command
% |\colophon| starts this part and prints some standard information
% about your fonts and \LaTeX{} package.  You may want to add some
% information after this command.
%
%
% \StopEventually{%
% \clearpage
% \bibliography{nostarch}
% \bibliographystyle{unsrt}}
% 
% \clearpage
%
%
%\section{Implementation}
%\label{sec:impl}
%
%\subsection{Identification}
%\label{sec:ident}
%
% We start with the declaration who we are.  Most |.dtx| files put
% driver code in a separate driver file |.drv|.  We roll this code into the
% main file, and use the pseudo-guard |<gobble>| for it.
%    \begin{macrocode}
%<class>\NeedsTeXFormat{LaTeX2e}
%<*gobble>
\ProvidesFile{nostarch.dtx}
%</gobble>
%<class>\ProvidesClass{nostarch}
%<nshyper>\ProvidesPackage{nshyper}
%<*!ist>
[2020/04/29 v1.4 Typesetting books for No Starch Press]
%</!ist>
%    \end{macrocode}
%
% And the driver code:
%    \begin{macrocode}
%<*gobble>
\documentclass{ltxdoc}
\usepackage{array}
\usepackage{url,amsfonts}
\usepackage[breaklinks,colorlinks,linkcolor=black,citecolor=black,
            pagecolor=black,urlcolor=black,hyperindex=false]{hyperref}
\PageIndex
\CodelineIndex
\RecordChanges
\EnableCrossrefs
\begin{document}
  \DocInput{nostarch.dtx}
\end{document}
%</gobble> 
%<*class>
%    \end{macrocode}
%   
%
%\subsection{Options}
%\label{sec:options}
%
%
% First, let us decide whether we have non-free fonts:
%    \begin{macrocode}
\newif\ifnostarch@cfonts
\nostarch@cfontstrue
\DeclareOption{cfonts}{\nostarch@cfontstrue}
\DeclareOption{nocfonts}{\nostarch@cfontsfalse}
%    \end{macrocode}
%
% The size-changing options produce a warning:
%    \begin{macrocode}
\long\def\nostarch@size@warning#1{%
  \ClassWarning{nostarch}{Size-changing option #1 will not be
    honored}}%
\DeclareOption{8pt}{\nostarch@size@warning{\CurrentOption}}%
\DeclareOption{9pt}{\nostarch@size@warning{\CurrentOption}}%
\DeclareOption{10pt}{\nostarch@size@warning{\CurrentOption}}%
\DeclareOption{11pt}{\nostarch@size@warning{\CurrentOption}}%
\DeclareOption{12pt}{\nostarch@size@warning{\CurrentOption}}%
%    \end{macrocode}
% 
%
%
% All other options are passed to \progname{book}:
%    \begin{macrocode}
\DeclareOption*{\PassOptionsToClass{\CurrentOption}{book}}
%    \end{macrocode}
% 
% Now we read the configuration file
%    \begin{macrocode}
\InputIfFileExists{nostarch.cfg}{%
  \ClassInfo{nostarch}{%
    Loading configuration file nostarch.cfg}}{%
  \ClassInfo{nostarch}{%
    Configuration file nostarch.cfg is not found}}
%    \end{macrocode}
% And process the options:
%    \begin{macrocode}
\ProcessOptions\relax
%    \end{macrocode}
%
%
% 
%
%
%\subsection{Loading Class and Packages}
%\label{sec:loading}
%
% We start with the base class
%    \begin{macrocode}
\LoadClass[10pt]{book}
%    \end{macrocode}
%
% A bunch of packages:
%    \begin{macrocode}
\RequirePackage{ifpdf, fancyhdr, fancyvrb, booktabs, graphicx,
  listings, caption, makeidx, upquote}
%    \end{macrocode}
%
% We do not want the option |document| with \progname{ragged2e}: this
% calls \progname{footmisc}, while we want our own footnotes:
%    \begin{macrocode}
\RequirePackage{ragged2e}
\setlength\RaggedRightParindent{\parindent}
\RaggedRight
%    \end{macrocode}
% 
%
% \begin{macro}{\@nschapterpreamble}
%   This is the preamble to bibliography and index commands used when
%   \progname{nshyper} is loaded.  Normally it is empty. 
%    \begin{macrocode}
\def\@nschapterpreamble{}
%    \end{macrocode}
%   
% \end{macro}
%
%
%\subsection{Fonts}
%\label{sec:fonts}
%
% If we have commercial fonts, we load them.  Note that the body text
% has roman font at 10pt, and typewriter at 8.5pt.
% Therefore we will load |thsmc| scaled.  We also add |dgdefault| for
% dogma family
%    \begin{macrocode}
\ifnostarch@cfonts
  \RequirePackage{nbaskerv}%
  \RequirePackage{futurans}%
  \RequirePackage[scaled=0.85]{thsmc}%
  \RequirePackage{dogma}%
  \newcommand{\dgdefault}{fdg}%
%    \end{macrocode}
% Otherwise we just use sans serif font for dogma.
%    \begin{macrocode}
\else
  \newcommand{\dgdefault}{\sfdefault}%
\fi
%    \end{macrocode}
%
%
% And switch to normal size---just in case
%    \begin{macrocode}
\normalsize
%    \end{macrocode}
% 
%
%
%\subsection{Page Dimensions and Paragraphing}
%\label{sec:page}
%
% \begin{macro}{\paperheight}
% \begin{macro}{\paperwidth}
% The trim size:
%    \begin{macrocode}
\setlength{\paperheight}{9.25in}
\setlength{\paperwidth}{7in}
%    \end{macrocode}
% \end{macro}
% \end{macro}
% \begin{macro}{\pdfpaperheight}
% \begin{macro}{\pdfpaperwidth}
% \begin{macro}{\pdfvorigin}
% \begin{macro}{\pdfhorigin}
% If we deal with \progname{pdftex}, we can use this information more
% creatively.  This was inspired by
% \progname{memoir}~\cite{Wilson04:Memoir}.
%    \begin{macrocode}
\ifpdf\relax
  \pdfpageheight=\paperheight
  \pdfpagewidth=\paperwidth
  \pdfvorigin=1in
  \pdfhorigin=1in
\fi
%    \end{macrocode}
% \end{macro}
% \end{macro}
% \end{macro}
% \end{macro}
%
% \begin{macro}{\topmargin}
% \changes{v0.2}{2008/03/03}{Moved down} 
% \changes{v0.7}{2008/04/04}{Changed dimensions} 
% \changes{v1.1}{2008/05/24}{Made shorter} 
%   The top margin is 0.625''.  We use fake headers of 12pt:
%    \begin{macrocode}
\setlength\topmargin{0.625in}
\addtolength\topmargin{-1in}
\addtolength\topmargin{-12pt}
%    \end{macrocode}  
% \end{macro}
% \begin{macro}{\textheight}
%   This is the height of the text including footnotes, but excluding
%   running head and foot. 
%    \begin{macrocode}
\setlength\textheight{8in}
\addtolength{\textheight}{-0.207in}
%    \end{macrocode}
% Now we take care of the first line height:
%    \begin{macrocode}
\addtolength\textheight{\topskip}
%    \end{macrocode}   
% \end{macro}
%
% \begin{macro}{\evensidemargin}
% \begin{macro}{\oddsidemargin}
%   The margins on even and odd pages are 0.687''+0.833'' (margin par
%   width) + 0.167'' (margin par gap):
%    \begin{macrocode}
\setlength\evensidemargin{0.687in}
\addtolength{\evensidemargin}{-1in}
\addtolength{\evensidemargin}{0.833in}
\addtolength{\evensidemargin}{0.167in}
\setlength\oddsidemargin{\evensidemargin}
%    \end{macrocode}
% \end{macro}
% \end{macro}
% \begin{macro}{\textwidth}
%   The type area is 5.625'', but this includes side gap:
%    \begin{macrocode}
\setlength\textwidth{5.625in}
\addtolength\textwidth{-0.833in}
\addtolength\textwidth{-0.167in}
%    \end{macrocode}
% \end{macro}
%
% \begin{macro}{\parindent}
%   The paragraph indentation is 0.25'':
%    \begin{macrocode}
\setlength\parindent{0.25in}
%    \end{macrocode}
% \end{macro}
%
%
%
% \begin{macro}{\headheight}
% \begin{macro}{\headsep}
%   We do not have headers in these books.  Fancyhdr sets |headheight|
%   to 12pt, so we compensate it above in |\topmargin|
%    \begin{macrocode}
\setlength\headheight{12pt}
\setlength\headsep{0pt}
%    \end{macrocode}
% \end{macro}
% \end{macro}
%
% \begin{macro}{\footskip}
% The footer is 8.792'' from top
%    \begin{macrocode}
\setlength{\footskip}{9in}
\addtolength{\footskip}{-\textheight}
\addtolength{\footskip}{0.067in}
\addtolength{\footskip}{\baselineskip}
\addtolength{\footskip}{-1in}
%    \end{macrocode}
% \end{macro}
%
% \begin{macro}{\widowpenalty}
% \begin{macro}{\clubpenalty}
% \changes{v0.3}{2008/03/11}{Changed settings} 
%   We forbid widows and clubs.  To prevent too extended pages we use
%   |\raggedbottom|:
%    \begin{macrocode}
\widowpenalty=10000
\clubpenalty=10000
\raggedbottom
%    \end{macrocode}
%   
% \end{macro}
% \end{macro}
% 
% 
%
%\subsection{Headers and Footers}
%\label{sec:headers}
%
%
% \begin{macro}{\headrulewidth}
% \begin{macro}{\footrulewidth}
%   We do not want decorative rules:
%    \begin{macrocode}
\renewcommand{\headrulewidth}{0pt}
\renewcommand{\footrulewidth}{0pt}
%    \end{macrocode}
% \end{macro}
% \end{macro}
% 
%
% We do not have headers:
%    \begin{macrocode}
\pagestyle{fancy}
\lhead{}
\rhead{}
\chead{}
%    \end{macrocode}
% 
% The right footer is stuck 0.833''+0.167'' = 1'' to the right
%    \begin{macrocode}
\fancyhfoffset[L]{1in}
%    \end{macrocode}
%
% On even pages we put page number and chapter title in footer. 
%    \begin{macrocode}
\fancyfoot[RO]{\fontfamily{\sfdefault}\fontsize{6pt}{6pt}%
  \fontseries{lq}\selectfont\rightmark%
  \hspace{2em}\fontseries{bc}\fontsize{9pt}{6pt}\selectfont\thepage}
\fancyfoot[LO]{}
\fancyfoot[CO]{}
%    \end{macrocode}
% 
% On even pages we put page number and chapter number in footer:
%    \begin{macrocode}
\fancyfoot[LE]{\fontfamily{\sfdefault}\fontseries{bc}\fontsize{9pt}{6pt}%
  \selectfont\thepage\hspace{2em}%
  \fontseries{lq}\fontsize{6pt}{6pt}\selectfont%
  \leftmark}
\fancyfoot[RE]{}
\fancyfoot[CE]{}
%    \end{macrocode}
% 
%
%\subsection{Front Matter}
%\label{sec:frontmatter}
%
%
% \begin{macro}{\subtitle}
% \changes{v0.5}{2008/03/19}{Introduced the macro} 
%   We use subtitle of the book\dots
%    \begin{macrocode}
\newcommand{\subtitle}[1]{\gdef\@subtitle{#1}}
%    \end{macrocode}
% \end{macro}
% \begin{macro}{\@subtitle}
% \changes{v0.5}{2008/03/19}{Introduced the macro} 
%   \dots With empty default:
%    \begin{macrocode}
\def\@subtitle{}
%    \end{macrocode}
% \end{macro}
%
% \begin{macro}{\nostarchlogo}
% \changes{v0.5}{2008/03/19}{Introduced the macro} 
%   The logotype of the publisher:
%    \begin{macrocode}
\newcommand{\nostarchlogo}[1]{\gdef\@nostarchlogo{#1}}
%    \end{macrocode}
% \end{macro}
% \begin{macro}{\@nostarchlogo}
% \changes{v0.5}{2008/03/19}{Introduced the macro} 
%   The default it empty:
%    \begin{macrocode}
\def\@nostarchlogo{}
%    \end{macrocode}
% \end{macro}
% \begin{macro}{\nostarchlocation}
% \changes{v0.5}{2008/03/19}{Introduced the macro} 
%   Similarly for location:
%    \begin{macrocode}
\newcommand{\nostarchlocation}[1]{\gdef\@nostarchlocation{#1}}
%    \end{macrocode}
% \end{macro}
% \begin{macro}{\@nostarchlocation}
% \changes{v0.5}{2008/03/19}{Introduced the macro} 
%   The default is SF:
%    \begin{macrocode}
\def\@nostarchlocation{San Francisco}
%    \end{macrocode}
% \end{macro}
%
%
% \begin{macro}{\makehalftitle}
% \changes{v0.5}{2008/03/19}{Introduced the macro} 
% \changes{v0.7}{2008/04/04}{Changed dimensions} 
%   Now we are ready to typeset the half title page.  It is flushed
%   right, cannot be to the left of the page center and must be 0.25''
%   indented from the right:
%    \begin{macrocode}
\def\makehalftitle{%
  \clearpage
  {\pagestyle{empty}\cleardoublepage}%
  \thispagestyle{empty}%
  \null\vskip1.025in%
  \leavevmode\noindent\hspace{1.66in}%
    \begin{minipage}[t]{2.5625in}%
    \flushright 
    \fontfamily{\sfdefault}\fontseries{b}%
    \fontsize{12\p@}{13\p@}\selectfont
    \MakeUppercase{\@title}%
  \end{minipage}}
%    \end{macrocode}
% \end{macro}
%
%
% \begin{macro}{maketitle}
% \changes{v0.5}{2008/03/19}{Changed LaTeX default} 
% \changes{v0.7}{2008/04/04}{Changed dimensions} 
% Now the main title page:
%    \begin{macrocode}
\def\maketitle{%
  \bgroup
  \clearpage
  {\pagestyle{empty}\cleardoublepage}%
  \thispagestyle{empty}%
  \bgroup
  \list{}{\leftmargin-1in\rightmargin\z@\labelwidth\z@%
    \labelsep\z@\listparindent\z@\parsep\z@\itemsep\z@%
    \topsep\z@\partopsep\z@}%
  \item\relax
  \centering
  \null\vskip0.855in%
  {\fontfamily{\dgdefault}\fontseries{bk}\fontsize{36\p@}{37\p@}\selectfont
    \MakeUppercase{\@title}\par}%
  \if\@subtitle\@empty\relax\else
     {\null\vskip25\p@%
     \fontfamily{\dgdefault}\fontseries{bq}\fontsize{24\p@}{25\p@}\selectfont
    \@subtitle\par}%
  \fi
  \vspace{1.7in}%
  \vspace{-\baselineskip}%
  {\fontfamily{\sfdefault}\fontseries{hq}\fontsize{18\p@}{22\p@}\selectfont
    by \@author\par}%
  \vspace{0.45in}%
  \vspace{\baselineskip}%
  \if\@nostarchlogo\@empty\relax\else
    \includegraphics[height=1.08in]{\@nostarchlogo}\par
  \fi
  \vspace{12\p@}%
  \normalfont\@nostarchlocation\par
  \endlist
  \egroup
  \clearpage\egroup}
%    \end{macrocode}
% \end{macro}
%
% \begin{macro}{copyrightpage}
% \changes{v0.5}{2008/03/19}{Introduced the macro} 
%   This is the page copyright data:
%    \begin{macrocode}
\newenvironment{copyrightpage}{%
  \clearpage
  \bgroup
  \list{}{\leftmargin-1in\rightmargin\z@\labelwidth\z@%
    \labelsep\z@\listparindent\z@\parsep8\p@\itemsep\z@%
    \topsep\z@\partopsep\z@}%
  \item\relax  
  \thispagestyle{empty}%
  \normalfont\fontsize{8\p@}{10\p@}\selectfont}{%
  \endlist
  \egroup
  \clearpage}
%    \end{macrocode}
% \end{macro}
%
% \begin{macro}{cipblock}
% \changes{v0.5}{2008/03/19}{Introduced the macro} 
%   CIP block is supposed to be in typewriter font.  So we use
%   verbatim:
%    \begin{macrocode}
\newenvironment{cipblock}{%
  \leavevmode\par\normalfont\fontsize{8\p@}{10\p@}\selectfont
  \textit{Library of Congress Cataloging-in-Publication Data}\par%
  \ttfamily\@vobeyspaces\obeylines
  \parindent\z@\parfillskip\@flushglue\parskip\z@skip}{\par\vskip8pt\relax}
%    \end{macrocode}   
% \end{macro}
%
%
% \begin{macro}{dedicationpage}
% \changes{v0.5}{2008/03/19}{Introduced the macro} 
%   Dedication is centered:
%    \begin{macrocode}
\newenvironment{dedicationpage}{%
  \clearpage
  {\pagestyle{empty}\cleardoublepage}%
  \thispagestyle{empty}%
  \null\vskip1.175in%
  \centering\normalfont}{\clearpage}
%    \end{macrocode}
% \end{macro}
%
%
%\subsection{Tables of Contents}
%\label{sec:toc}
%
% Standard \LaTeX{} has just one table of contents.  We have two.  The
% table of brief contents will have the extension |tbc|, while the
% standard table of contents will have the standard extension |toc|. 
%
% \begin{macro}{\briefcontentsname}
%   The name for brief contents
%    \begin{macrocode}
\def\briefcontentsname{Brief Contents}
%    \end{macrocode}
%   
% \end{macro}
%
% \begin{macro}{\contentsname}
%   Same with |\contentsname|
%    \begin{macrocode}
\def\contentsname{Contents in Detail}%
%    \end{macrocode}
%  Special treatment if \progname{babel} is used:
%    \begin{macrocode}
\AtBeginDocument{\@ifpackageloaded{babel}{%
  \addto\captionsenglish{%
    \renewcommand{\contentsname}%
    {Contents in Detail}}}{}}
%    \end{macrocode}
%  
% \end{macro}
%
%
% \begin{macro}{\brieftableofcontents}
% \changes{v0.5}{2008/03/19}{Introduced macro} 
% \changes{v0.7}{2008/04/04}{Changed dimensions} 
% \changes{v0.8}{2008/04/07}{Changed dimensions} 
% \changes{v1.1}{2008/05/24}{Added toc name to the verso footers} 
%   This is the new macro:
%    \begin{macrocode}
\newcommand\brieftableofcontents{%
  \clearpage
  {\pagestyle{empty}\cleardoublepage}%
  \bgroup
  \thispagestyle{empty}%
  \markboth{\briefcontentsname}{\briefcontentsname}%
  \null\vskip0.925in%
  \list{}{\leftmargin-1.0in\rightmargin0.4in\labelwidth\z@%
    \labelsep\z@\listparindent\z@\parsep\z@\itemsep\z@%
    \topsep\z@\partopsep\z@}%
  \item\relax  
    {\centering
      \fontfamily{\dgdefault}\fontseries{bs}\fontsize{16\p@}{18\p@}\selectfont
      \MakeUppercase{\briefcontentsname}\par}\vskip40pt%
  \@starttoc{tbc}%
  \endlist
  \egroup}
%    \end{macrocode}
%   
% \end{macro}
% \begin{macro}{\tableofcontents}
% \changes{v0.5}{2008/03/19}{Changed LaTeX default} 
% \changes{v0.7}{2008/04/04}{Changed dimensions} 
% \changes{v0.8}{2008/04/07}{Changed dimensions} 
% \changes{v1.1}{2008/05/24}{Added toc name to the verso footers} 
% \changes{v1.2}{2008/06/01}{Added toc name to the verso footers} 
%   This is the new macro:
%    \begin{macrocode}
\renewcommand\tableofcontents{%
  \clearpage
  {\pagestyle{empty}\cleardoublepage}%
  \bgroup
  \thispagestyle{empty}%
  \markboth{\contentsname}{\contentsname}%
  \null\vskip0.175in%
  \list{}{\leftmargin-1.0in\rightmargin0.6in\labelwidth\z@%
    \labelsep\z@\listparindent\z@\parsep\z@\itemsep\z@%
    \topsep\z@\partopsep\z@}%
  \item\relax  
    {\centering
      \fontfamily{\dgdefault}\fontseries{bs}\fontsize{16\p@}{18\p@}\selectfont
      \MakeUppercase{\contentsname}\par}\vskip18pt%
  \@starttoc{toc}%
  \endlist
  \egroup}
%    \end{macrocode}
% \end{macro}
%
% \begin{macro}{\@dotsep}
% \changes{v0.7}{2008/04/04}{Changed LaTeX default} 
% \changes{v0.8}{2008/04/07}{Tightened dots} 
%  Standard \LaTeX{} provides 4.5 for the distance between dots.  We
%  will make this tighter:
%    \begin{macrocode}
\renewcommand\@dotsep{1.5}
%    \end{macrocode}
%  
% \end{macro}
%
% \begin{macro}{\l@part}
% \changes{v0.5}{2008/03/19}{Changed LaTeX default} 
%   This is how we typeset parts in the table of contents
%    \begin{macrocode}
\renewcommand*\l@part[2]{%
  \ifnum \c@tocdepth >-2\relax
    \addpenalty{-\@highpenalty}%
    \addvspace{28\p@}%
    \begingroup
      \parindent \z@ \rightskip \@pnumwidth
      \parfillskip -\@pnumwidth
      {\leavevmode
       \fontfamily{\sfdefault}\fontseries{bq}\fontsize{12\p@}{14\p@}\selectfont
       #1}\par
       \nobreak
         \global\@nobreaktrue
         \everypar{\global\@nobreakfalse\everypar{}}%
    \endgroup
  \fi}
%    \end{macrocode}   
% \end{macro}
% \begin{macro}{\l@chapter}
% \changes{v0.5}{2008/03/19}{Changed LaTeX default} 
%   This is how we typeset chapters in the table of contents
%    \begin{macrocode}
\renewcommand*\l@chapter[2]{%
  \ifnum \c@tocdepth >\m@ne
    \addpenalty{-\@highpenalty}%
    \vskip 20\p@
    \setlength\@tempdima{1.5em}%
    \begingroup
      \parindent \z@ \rightskip \@pnumwidth
      \parfillskip -\@pnumwidth
      \leavevmode 
      \fontfamily{\sfdefault}\fontseries{bq}\fontsize{10\p@}{12\p@}\selectfont
      #1\nobreak\hfil \nobreak\hb@xt@\@pnumwidth{\hss #2}\par
      \vskip4\p@\penalty\@highpenalty
    \endgroup
  \fi}
%    \end{macrocode}
% \end{macro}
%
% \begin{macro}{\sectionnumberwidth}
% \changes{v1.3}{2008/06/06}{Introduced the length} 
%   The number width for sections.  Normally the width of |99.99|
%    \begin{macrocode}
\newlength{\sectionnumberwidth}
\settowidth{\sectionnumberwidth}{%
  \fontfamily{\sfdefault}\fontseries{k}\fontsize{9\p@}{11\p@}\selectfont
  99.99}
%    \end{macrocode}   
% \end{macro}
% \begin{macro}{\subsectionnumberwidth}
% \changes{v1.3}{2008/06/06}{Introduced the length} 
%   The number width for subsections.  Normally the width of
%   |99.99.99| 
%    \begin{macrocode}
\newlength{\subsectionnumberwidth}
\settowidth{\subsectionnumberwidth}{%
  \fontfamily{\sfdefault}\fontseries{k}\fontsize{9\p@}{11\p@}\selectfont
  99.99.99}
%    \end{macrocode}   
% \end{macro}
%
% \begin{macro}{\l@section}
% \changes{v0.5}{2008/03/19}{Changed LaTeX default} 
% \changes{v1.3}{2008/06/06}{Made number width user settable} 
%   This is how we typeset sections in the table of contents
%    \begin{macrocode}
\renewcommand{\l@section}[2]{\@dottedtocline{1}{\z@}{%
    \sectionnumberwidth}{%
    \fontfamily{\sfdefault}\fontseries{k}\fontsize{9\p@}{11\p@}\selectfont#1}{%
    \fontfamily{\sfdefault}\fontseries{k}\fontsize{9\p@}{11\p@}\selectfont#2}}
%    \end{macrocode}
% \end{macro}
%
% \begin{macro}{\l@subsection}
% \changes{v0.5}{2008/03/19}{Changed LaTeX default} 
% \changes{v1.1}{2008/05/24}{Changed numwidth} 
% \changes{v1.2}{2008/06/01}{Changed numwidth again} 
% \changes{v1.3}{2008/06/06}{Made number width user settable} 
%   This is how we typeset subsections in the table of contents
%    \begin{macrocode}
\renewcommand{\l@subsection}[2]{\@dottedtocline{2}{0.5in}%
  {\subsectionnumberwidth}{%
    \fontfamily{\sfdefault}\fontseries{k}\fontsize{9\p@}{11\p@}\selectfont#1}{%
    \fontfamily{\sfdefault}\fontseries{k}\fontsize{9\p@}{11\p@}\selectfont#2}}
%    \end{macrocode}
% \end{macro}
%
%
%
% \begin{macro}{\l@bpart}
% \changes{v0.6}{2008/03/21}{Introduced macro} 
%   There is a special formatting of parts in the brief table of
%   contents: 
%    \begin{macrocode}
\newcommand*\l@bpart[2]{%
  \ifnum \c@tocdepth >-2\relax
    \addpenalty{-\@highpenalty}%
    \addvspace{14\p@}%
    \begingroup
      \parindent \z@ \rightskip \@pnumwidth
      \parfillskip -\@pnumwidth
      {\leavevmode
       \fontfamily{\sfdefault}\fontseries{h}\fontsize{9\p@}{14\p@}\selectfont
       #1}\par\vskip7\p@%
       \nobreak
         \global\@nobreaktrue
         \everypar{\global\@nobreakfalse\everypar{}}%
    \endgroup
  \fi}
%    \end{macrocode}
% \end{macro}
%
% \begin{macro}{\l@bchapter}
% \changes{v0.6}{2008/03/21}{Introduced macro} 
%   There is a special formatting of chapters in the brief table of
%   contents: 
%    \begin{macrocode}
\newcommand*\l@bchapter[2]{%
  \ifnum \c@tocdepth >\m@ne
    \addpenalty{-\@highpenalty}%
    \setlength\@tempdima{1.5em}%
    \begingroup
      \parindent \z@ \rightskip \@pnumwidth
      \parfillskip -\@pnumwidth
      \leavevmode 
      \fontfamily{\sfdefault}\fontseries{k}\fontsize{9\p@}{14\p@}\selectfont
      #1\nobreak\leaders\hbox{$\m@th
        \mkern \@dotsep mu\hbox{.}\mkern \@dotsep
        mu$}\hfill
      \nobreak\hb@xt@\@pnumwidth{\hss #2}\par
      \penalty\@highpenalty
    \endgroup
  \fi}
%    \end{macrocode}
% \end{macro}
%
%
%\subsection{Main Matter}
%\label{sec:mainmatter}
%
% \begin{macro}{\mainmatter}
% \changes{v0.6}{2008/03/19}{Changed LaTeX default} 
%   We redefine |\mainmatter| to take care of that blank verso
%   page\dots
%    \begin{macrocode}
\renewcommand\mainmatter{%
  \clearpage
  {\thispagestyle{empty}\cleardoublepage}%
  \@mainmattertrue
  \pagenumbering{arabic}}
%    \end{macrocode}
% \end{macro}
%
%
%\subsection{Back Matter}
%\label{sec:backmatter}
%
% \begin{macro}{\backmatter}
% \changes{v0.6}{2008/03/21}{Changed LaTeX default} 
%   We redefine |\backmatter| to take care of that blank verso
%   page\dots
%    \begin{macrocode}
\renewcommand\backmatter{%
  \clearpage
  {\thispagestyle{empty}\cleardoublepage}%
  \@mainmatterfalse}
%    \end{macrocode}
% \end{macro}
%
%
%\subsection{Sectioning}
%\label{sec:sectioning}
%
% We do not number sections and below:
%    \begin{macrocode}
\setcounter{secnumdepth}{0}
%    \end{macrocode}
% 
% \begin{macro}{\sectionmark}
%   Our section commands do not mark:
%    \begin{macrocode}
\def\sectionmark#1{}%
%    \end{macrocode}   
% \end{macro}
%
%
% Chapters, unlike the ones in~\cite{classes}, have \emph{two} obligatory
% arguments. The second argument is the command to fill the ``circular
% graphics''. Otherwise we follow the design in~\cite{classes}.  One
% problem, however: we \emph{cannot} use |\secdef| because we have
% three arguments, not two.
%
%
% \begin{macro}{\if@firstpara}
%   We introduce a macro to get special typesetting for the first
%   paragraph after a part or a chapter begins:
%    \begin{macrocode}
\newif\if@firstpara
\@firstparafalse
%    \end{macrocode}   
% \end{macro}
%
% \begin{macro}{\@chapterart}
%   We also keep track of the current artwork for chapter start:
%    \begin{macrocode}
\def\@chapterart{}
%    \end{macrocode}
% \end{macro}
%
% \begin{macro}{\secdef}
% \changes{v0.6}{2008/03/21}{Changed LaTeX default} 
%   The standard \LaTeX{} |secdef| assumes that starred commands have no
%   optional argument.  This is not right for us:
%    \begin{macrocode}
\def\secdef#1#2{\@ifstar{\@dblarg{#2}}{\@dblarg{#1}}}
%    \end{macrocode}
%   
% \end{macro}
%
%
% \begin{macro}{\part}
% \changes{v0.5}{2008/03/16}{Changed LaTeX default} 
%   Our parts differ from the ones in standard LaTeX since we use empty
%   pagestyle rather than plain in part pages.  The trick with
%   |\cleardoublepage| is from~\cite{Oostrum04:Fancyhdr}
%    \begin{macrocode}
\renewcommand\part{%
  \clearpage
  {\pagestyle{empty}\cleardoublepage}%
  \everypar{}%
  \markboth{}{}%
  \def\@chapterart{}%
  \thispagestyle{empty}%
  \null\vskip0.715in%
  \secdef\@part\@spart}
%    \end{macrocode} 
% \end{macro}
% 
% \begin{macro}{\@part}
% \changes{v0.5}{2008/03/16}{Changed LaTeX default} 
% \changes{v0.7}{2008/04/04}{Changed dimensions} 
%   This code is used when we have a numbered part.
%    \begin{macrocode}
\def\@part[#1]#2{%
    \ifnum \c@secnumdepth >-2\relax
      \refstepcounter{part}%
      \addcontentsline{toc}{part}{\MakeUppercase{\partname} 
        \thepart\protect\linebreak\penalty\@highpenalty\MakeUppercase{#1}}%
      \addcontentsline{tbc}{bpart}{\MakeUppercase{\partname~\thepart: #1}}%
    \else
      \addcontentsline{toc}{part}{\MakeUppercase{#1}}%
      \addcontentsline{tbc}{bpart}{\MakeUppercase{#1}}%
    \fi
    {\centering
     \interlinepenalty \@M
     \normalfont
     \ifnum \c@secnumdepth >-2\relax
       \fontfamily{\sfdefault}\fontsize{80pt}{80pt}\fontseries{bc}\selectfont
       \MakeUppercase{\partname\nobreakspace\thepart}%
       \par
       \vskip 0.45in%
     \fi
     \fontfamily{\dgdefault}\fontsize{16pt}{16pt}\fontseries{br}\selectfont
     \MakeUppercase{#2}\par}%
    \@endpart}
%    \end{macrocode}
% \end{macro}
%
% \begin{macro}{\@spart}
% \changes{v0.5}{2008/03/16}{Changed LaTeX default} 
%   This is used for starred parts (probably not something we are
%   going to encounter\dots)
%    \begin{macrocode}
\def\@spart[#1]#2{%
  \addcontentsline{toc}{part}{\MakeUppercase{#1}}%
  \addcontentsline{tbc}{bpart}{\MakeUppercase{#1}}%
    {\centering
      \fontfamily{\dgdefault}\fontsize{16pt}{16pt}\fontseries{br}\selectfont
     \MakeUppercase{#2}\par}%
    \@endpart}
%    \end{macrocode}
% \end{macro}
%
% \begin{macro}{\@endpart}
% \changes{v0.5}{2008/03/16}{Changed LaTeX default} 
%   When part is finished, we change the settings of the first paragraph:
%    \begin{macrocode}
\def\@endpart{%
  \@afterindentfalse
  \vskip49\p@%
   \@firstparatrue
   \@afterheading}    
%    \end{macrocode}
%   
% \end{macro}
%
% \begin{macro}{\chapter}
% \changes{v0.6}{2008/03/21}{Reverted to standard interface} 
% First, we suppress floats, set up empty page style and delete
% indentation after the chapter.  Then we check whether the chapter is
% starred:
%    \begin{macrocode}
\def\chapter{%
  \clearpage
  {\pagestyle{empty}\cleardoublepage}%
  \everypar{}%
  \thispagestyle{empty}%
  \global\@topnum\z@
  \@afterindentfalse
  \gdef\@chapterart{}%
  \secdef\@chapter\@schapter}
%    \end{macrocode}
% \end{macro}
%
%
% \begin{macro}{\@chapter}
% \changes{v0.2}{2008/03/04}{Moved up} 
% \changes{v0.6}{2008/03/21}{Reverted to standard interface} 
% \changes{v1.3}{2008/06/06}{Corrected a stupid mistake: put
% refstepcounter out of a group}
%   The actual work is done by |\@chapter| macro.
%    \begin{macrocode}
\def\@chapter[#1]#2{%
  \vspace*{0.622in}
  \ifnum \c@secnumdepth > \m@ne
  \refstepcounter{chapter}%
  \typeout{Chapter:\thechapter\space\@currentlabel}%
  \markboth{\@chapapp~\thechapter}{#1}%
  \typeout{\@chapapp\space\thechapter.}%
  \addcontentsline{toc}{chapter}{\thechapter%
    \protect\linebreak\penalty\@highpenalty
    \MakeUppercase{#1}}%
  \addcontentsline{tbc}{bchapter}{%
    \chaptername~\thechapter:~#1}%
  {\centering
    \fontfamily{\sfdefault}\fontseries{bc}\fontsize{120pt}{120pt}
    \selectfont
    \thechapter\par\nobreak\vskip27pt}%
  \else
  \addcontentsline{toc}{chapter}{%
    \MakeUppercase{#1}}%
  \addcontentsline{tbc}{bchapter}{#1}%
  \markboth{#1}{#1}%
  \fi
  {\centering
    \fontfamily{\dgdefault}\fontseries{br}\fontsize{16pt}{20pt}\selectfont
    \MakeUppercase{#2}\par}\nobreak\vskip25pt\relax%
  \@firstparatrue
  \@afterheading}%
%    \end{macrocode}
% \end{macro}
%
% \begin{macro}{\@schapter}
% \changes{v0.6}{2008/03/21}{Reverted to standard interface} 
% \changes{v0.7}{2008/04/04}{Changed dimensions} 
%   This is used when we have a starred chapter:
%    \begin{macrocode}
\def\@schapter[#1]#2{%
  \vspace*{2.747in}
  {\centering
    \addcontentsline{toc}{chapter}{%
      \MakeUppercase{#1}}%
    \addcontentsline{tbc}{bchapter}{#1}%
    \markboth{#1}{#1}%
    {\fontfamily{\dgdefault}\fontseries{br}\fontsize{16pt}{20pt}\selectfont
      \MakeUppercase{#2}\par}}\nobreak\vskip25pt\relax%
   \@firstparatrue
   \@afterheading}%
%    \end{macrocode}
% \end{macro}
%
% \begin{macro}{\@afterheading}
% \changes{v0.3}{2008/03/12}{Moved the circle art} 
% \changes{v0.4}{2008/03/14}{Added special treatment for the case of
% empty chapter art} 
% \changes{v0.7}{2008/04/04}{Changed dimensions} 
% We cannot use the standard |\@afterheading| since we want to put the
% first paragraph in larger font and put there the artwork.  So we
% take |\@afterheading| code and patch it:
%    \begin{macrocode}
\def\@afterheading{%
  \@nobreaktrue
  \everypar{%
    \if@nobreak
      \@nobreakfalse
      \clubpenalty \@M
       \setbox\z@\lastbox
    \else
      \clubpenalty \@clubpenalty
      \everypar{}%
    \fi
      \if@firstpara
%    \end{macrocode}
% The first paragraph has a special font and artwork in the beginning:
%    \begin{macrocode}
       \rule{\z@}{49\p@}%
       \ifx\@chapterart\@empty%
       \else
         \makebox[0pt][r]{\raisebox{-0.5in}[0pt][0pt]{%
            \@chapterart\hspace{0.21in}}}%
        \parshape=5 
        0.963in 3.622in 
        0.913in 3.672in 
        0.813in 3.772in 
        0.563in 4.022in 
        0in \textwidth 
      \fi
      \fontsize{14pt}{16.5pt}\selectfont%
      \parskip=3pt
      \else
        \parskip=0pt
        \normalsize\selectfont
       \fi
    \@firstparafalse}}%
%    \end{macrocode}
% \end{macro}
%
% \begin{macro}{\chapterart}
% \changes{v0.6}{2008/03/21}{Introduced the command} 
%   This creates the chapterart:
%    \begin{macrocode}
\newcommand{\chapterart}[1]{\gdef\@chapterart{#1}}%
%    \end{macrocode}   
% \end{macro}
%
% \begin{macro}{\chapterartfile}
% \changes{v0.6}{2008/03/21}{Introduced the command} 
%   The simplified version of |\chapterart|:
%    \begin{macrocode}
\newcommand{\chapterartfile}[1]{\gdef\@chapterart{%
    \includegraphics[width=1.264in]{#1}}}%
%    \end{macrocode}
%   
% \end{macro}
%
%
% \begin{macro}{\section}
% \changes{v0.3}{2008/03/09}{Changed indentation}
%   Sections correspond to |HeadA| in our specs.  We use |\everypar{}|
%   in case section starts just after chapter start.
%    \begin{macrocode}
\renewcommand{\section}{\everypar{}%
  \@startsection{section}{1}{-0.75in}{16pt}{6pt}{%
    \fontfamily{\sfdefault}\fontseries{b}\fontsize{12pt}{16pt}\selectfont}}%
%    \end{macrocode}   
% \end{macro}
% 
% \begin{macro}{\subsection}
%   Subsections correspond to |HeadB| in our specs:
%    \begin{macrocode}
\renewcommand{\subsection}{\everypar{}%
  \@startsection{subsection}{2}{0pt}{14pt}{4pt}{%
    \fontfamily{\sfdefault}\fontshape{it}%
    \fontseries{bc}\fontsize{12pt}{17pt}\selectfont}}%
%    \end{macrocode}   
% \end{macro}
%
%
% \begin{macro}{\subsubsection}
%   Subsubsections correspond to |HeadC| in our specs:
%    \begin{macrocode}
\renewcommand{\subsubsection}{\everypar{}%
  \@startsection{subsubsection}{3}{0pt}{10pt}{2pt}{%
  \fontfamily{\sfdefault}\fontseries{bp}\fontsize{10pt}{14pt}%
    \selectfont}}%
%    \end{macrocode}   
% \end{macro}
%
% We do not redefine levels below, leaving the options
% from~\cite{classes} in place.
%
%
%
%\subsection{Lists}
%\label{sec:lists}
%
% This follows design of~\cite{classes} with the values from our
% specifications:
% \begin{macro}{\leftmargin}
% \begin{macro}{\leftmargini}
% \begin{macro}{\leftmarginii}
% \begin{macro}{\leftmarginiii}
% \begin{macro}{\leftmarginiv}
%   The host of |\leftmargin| commands:
%    \begin{macrocode}
\setlength{\leftmargin}{0.25in}
\setlength{\leftmargini}{0.25in}
\setlength{\leftmarginii}{0.25in}
\setlength{\leftmarginiii}{0.25in}
\setlength{\leftmarginiv}{0.25in}
%    \end{macrocode}   
% \end{macro}
% \end{macro}
% \end{macro}
% \end{macro}
% \end{macro}
% 
% \begin{macro}{\rightmargin}
%   Right margin is always zero:
%    \begin{macrocode}
\setlength{\rightmargin}{0pt}
%    \end{macrocode}
% \end{macro}
% 
%
% \begin{macro}{\@listi}
% \begin{macro}{\@listI}
%   This is the default list on the first level
%    \begin{macrocode}
\def\@listi{\leftmargin\leftmargini
  \labelwidth\leftmargin\relax
  \advance\labelwidth-\labelsep\relax
  \parsep0\p@\relax
  \topsep6\p@\relax
  \itemsep4\p@\relax}
\let\@listI\@listi
%    \end{macrocode}
% \end{macro}
% \end{macro}
%
% \begin{macro}{\@listii}
% \begin{macro}{\@listiii}
% \begin{macro}{\@listiv}
% \begin{macro}{\@listv}
% \begin{macro}{\@listvi}
%   And the lists for the next levels:
%    \begin{macrocode}
\def\@listii{\leftmargin\leftmarginii
  \labelwidth\leftmarginii
  \advance\labelwidth-\labelsep
  \parsep0\p@\relax
  \topsep0\p@\relax
  \itemsep0\p@\relax}
\def\@listiii{\leftmargin\leftmarginiii
  \labelwidth\leftmarginiii
  \advance\labelwidth-\labelsep
  \parsep0\p@\relax
  \topsep0\p@\relax
  \itemsep0\p@\relax}
\def\@listiv{\leftmargin\leftmarginiv
  \labelwidth\leftmarginiv
  \advance\labelwidth-\labelsep
  \parsep0\p@\relax
  \topsep0\p@\relax
  \itemsep0\p@\relax}
\def\@listv{\leftmargin\leftmarginv
  \labelwidth\leftmarginv
  \advance\labelwidth-\labelsep
  \parsep0\p@\relax
  \topsep0\p@\relax
  \itemsep0\p@\relax}
\def\@listvi{\leftmargin\leftmarginvi
  \labelwidth\leftmarginvi
  \advance\labelwidth-\labelsep
  \parsep0\p@\relax
  \topsep0\p@\relax
  \itemsep0\p@\relax}
%    \end{macrocode}   
% \end{macro}
% \end{macro}
% \end{macro}
% \end{macro}
% \end{macro}
%
% \begin{macro}{itemize}
% \changes{v0.2}{2008/03/03}{Flush left bullets} 
%   We make bullets flushleft:
%    \begin{macrocode}
\def\itemize{%
  \ifnum \@itemdepth >\thr@@\@toodeep\else
    \advance\@itemdepth\@ne
    \edef\@itemitem{labelitem\romannumeral\the\@itemdepth}%
    \expandafter
    \list
      \csname\@itemitem\endcsname
      {\def\makelabel##1{##1}}%
  \fi}
%    \end{macrocode}
%   
% \end{macro}
% 
% \begin{macro}{enumerate}
% \changes{v0.2}{2008/03/03}{Flush left numbers} 
%   We make numbering flushleft:
%    \begin{macrocode}
\def\enumerate{%
  \ifnum \@enumdepth >\thr@@\@toodeep\else
    \advance\@enumdepth\@ne
    \edef\@enumctr{enum\romannumeral\the\@enumdepth}%
      \expandafter
      \list
        \csname label\@enumctr\endcsname
        {\usecounter\@enumctr\def\makelabel##1{##1}}%
        \fi}
%    \end{macrocode}
%   
% \end{macro}
% 
%
% \begin{macro}{quotation}
%   In |quotation| environment the paragraphs are indented.  We use
%   9pt Roman for quotations:
%    \begin{macrocode}
\renewenvironment{quotation}{%
  \list{}{\listparindent\parindent\relax
    \itemindent\listparindent\relax
    \rightmargin0.5in\relax
    \leftmargin0.5in\relax}%
    \item\fontsize{9pt}{11pt}\selectfont}{\endlist}
%    \end{macrocode}
% \end{macro}
%
%
% \begin{macro}{quote}
%   Quote is for short quotations without indentations:
%    \begin{macrocode}
\renewenvironment{quote}{%
  \list{}{\listparindent0pt\relax
    \itemindent\listparindent\relax
    \rightmargin0.5in\relax
    \leftmargin0.5in\relax}%
    \item\fontsize{9pt}{11pt}\selectfont}{\endlist}
%    \end{macrocode}   
% \end{macro}
%
% \begin{macro}{note}
%   Note is a special environment for asides.  It is in italics with
%   the word ``NOTE'' on the margin:
%    \begin{macrocode}
\newenvironment{note}{%  
  \list{\makebox[0pt][r]{\fontfamily{%
          \dgdefault}\fontseries{b}\fontsize{9pt}{11pt}\selectfont 
        NOTE\hspace{2em}}}{\listparindent0pt\relax
    \topsep9\p@\relax
    \itemindent0\p@\relax
    \rightmargin0\p@\relax
    \leftmargin0\p@\relax
    \labelwidth0\p@\relax
    \labelsep0\p@}%
    \item\itshape}{\vspace{-3pt}\endlist}
%    \end{macrocode}
% \end{macro}
%
%
%
%\subsection{Footnotes}
%\label{sec:footnotes}
%
% \begin{macro}{\footnoterule}
% \changes{v0.3}{2008/03/11}{Increased kern above the rule} 
% \changes{v1.1}{2008/06/01}{Increased kern below the rule} 
%   We want one inch by 0.25 footnote rule:
%    \begin{macrocode}
\renewcommand\footnoterule{%
  \kern6\p@
  \hrule height 0.25pt depth 0pt width 1in 
  \kern2\p@}
%    \end{macrocode}
% \end{macro}
% 
% \begin{macro}{\@footnotetext}
% \changes{v0.3}{2008/03/11}{Changed LaTeX default} 
%   We want 4pt between footnotes.  Unfortunately the standard
%   functions introduces |\@finalstrut|, which we do not need
%   here\dots 
%    \begin{macrocode}
\long\def\@footnotetext#1{\insert\footins{%
    \reset@font\footnotesize
    \interlinepenalty\interfootnotelinepenalty
    \splittopskip\footnotesep
    \splitmaxdepth \dp\strutbox \floatingpenalty \@MM
    \hsize\columnwidth \@parboxrestore
    \protected@edef\@currentlabel{%
       \csname p@footnote\endcsname\@thefnmark
    }%
    \color@begingroup
      \@makefntext{%
        \rule\z@{13.5pt}\ignorespaces#1}%
    \color@endgroup}}%
%    \end{macrocode}
% \end{macro}
%
% \begin{macro}{\footins}
% \changes{v0.3}{2008/03/11}{Changed LaTeX defaults} 
%   We make some space between footnotes:
%    \begin{macrocode}
\setlength{\skip\footins}{16\p@ \@plus 4\p@ \@minus 0\p@}
%    \end{macrocode}
%   
% \end{macro}
%
%
% \begin{macro}{\@makefntext}
% \changes{v1.1}{2008/05/24}{Added space between footnote mark and
% footnote text} 
%   We do not indent the footnotes:
%    \begin{macrocode}
\renewcommand\@makefntext[1]{%
    \parindent 0\p@%
    \RaggedRightParindent0\p@%
    \noindent
    \@makefnmark\space#1}
%    \end{macrocode}
% \end{macro}
%
%
%\subsection{Tables,  Figures and Listings}
%\label{sec:tbl_fig}
%
% \begin{macro}{\thefigure}
%   We want dash instead of dot between chapter number and figure
%   number:
%    \begin{macrocode}
\renewcommand \thefigure
     {\ifnum \c@chapter>\z@ \thechapter-\fi \@arabic\c@figure}
%    \end{macrocode}
% \end{macro}
% \begin{macro}{\thetable}
%   We want dash instead of dot between chapter number and table
%   number:
%    \begin{macrocode}
\renewcommand \thetable
     {\ifnum \c@chapter>\z@ \thechapter-\fi \@arabic\c@table}
%    \end{macrocode}
% \end{macro}
% \begin{macro}{\thelistings}
% \changes{v1.4}{2020/04/24}{Added \cs{AtBeginDocument}} 
%   And the same for listings.  Note that listings package uses
%   \cs{AtBeginDocument}, so we need to postpone this
%    \begin{macrocode}
\AtBeginDocument{%
  \renewcommand \thelstlisting
    {\ifnum \c@chapter>\z@ \thechapter-\fi \@arabic\c@lstlisting}}
%    \end{macrocode}
% \end{macro}
% 
%
% \begin{macro}{nostarchfigfmt}
%   This is our captions format for figures and listings
% \changes{v0.3}{2008/03/11}{Added style} 
%    \begin{macrocode}
\DeclareCaptionFormat{nostarchfigfmt}{\fontfamily{\sfdefault}%
  \fontshape{it}\fontsize{8.5pt}{9pt}\fontseries{k}\selectfont\unskip#1#2#3}
%    \end{macrocode}
% \end{macro}
% \begin{macro}{nostarchtabfmt}
% \changes{v0.3}{2008/03/11}{Added style} 
%   This is our captions format for tables
%    \begin{macrocode}
\DeclareCaptionFormat{nostarchtabfmt}{\fontfamily{\sfdefault}%
  \fontshape{rm}\fontsize{8.5pt}{9pt}\fontseries{h}\selectfont\unskip#1#2%
  \fontseries{k}\selectfont#3}
%    \end{macrocode}
% \end{macro}
%  
%
%
% The figure, listing and and table styles:
%    \begin{macrocode}
\captionsetup[figure]{format=nostarchfigfmt,singlelinecheck=off,
  aboveskip=8pt,belowskip=8pt}
\captionsetup[lstlisting]{format=nostarchfigfmt,singlelinecheck=off,
  aboveskip=8pt,belowskip=8pt}
\captionsetup[table]{format=nostarchtabfmt,singlelinecheck=off,
  aboveskip=0pt,belowskip=4pt}
%    \end{macrocode}
%
% \begin{macro}{\tbfont}
%   This is the font used for table body:
%    \begin{macrocode}
\def\tbfont{%
  \fontfamily{\sfdefault}\fontseries{k}\fontsize{8pt}{10pt}\selectfont}
%    \end{macrocode} 
% \end{macro}
%
% \begin{macro}{\thfont}
%   Font for table headers:
%    \begin{macrocode}
\def\thfont{%
  \fontfamily{\sfdefault}\fontseries{h}\fontsize{8pt}{10pt}\selectfont}
%    \end{macrocode}
% \end{macro}
%
%
% \begin{macro}{\heawyrulewidth}
% \begin{macro}{\lightrulewidth}
% \begin{macro}{\cmidrulewidth}
%   Changing \progname{booktabs} defaults:
%    \begin{macrocode}
\heavyrulewidth=3\p@
\lightrulewidth=1.5\p@
\cmidrulewidth=1.5\p@
%    \end{macrocode}
% \end{macro}
% \end{macro}
% \end{macro}
%
%
% \begin{macro}{\bottomrule}
%   Our |\bottomrule| is thin:
%    \begin{macrocode}
\def\bottomrule{\noalign{\ifnum0=`}\fi
  \@aboverulesep=\aboverulesep
  \global\@belowrulesep=\belowbottomsep
  \global\@thisruleclass=\@ne
  \@ifnextchar[{\@BTrule}{\@BTrule[\lightrulewidth]}}
%    \end{macrocode}
% \end{macro}
%
%
% \begin{macro}{\belowrulesep}
% \begin{macro}{\belowbottomsep}
% \begin{macro}{\abovetopsep}
%   Again redefining \progname{booktabs}:
%    \begin{macrocode}
\belowrulesep=0.7ex
\belowbottomsep=0.65pt
\aboverulesep=0.7ex
\abovetopsep=0.65pt
%    \end{macrocode}
%   
% \end{macro}
% \end{macro}
% \end{macro}
%  
%
%
%\subsection{Verbatim and Listings Customization}
%\label{sec:verbatim}
%
% \begin{macro}{Code}
%   This is for framed code:
%    \begin{macrocode}
\DefineVerbatimEnvironment{Code}{Verbatim}{frame=lines,framerule=0.25pt}
%    \end{macrocode}   
% \end{macro}
%
%
% \begin{macro}{\FV@ListParameterHook}
%   The skip before code is too large for us
%    \begin{macrocode}
\FV@AddToHook\FV@ListParameterHook{\vspace{-6pt}}
%    \end{macrocode}   
% \end{macro}
%
% We also enable verbatim footnotes.
%    \begin{macrocode}
\VerbatimFootnotes
%    \end{macrocode}
% 
% We set up captions with \progname{captions}:
%    \begin{macrocode}
\lstset{captionpos=b}
%    \end{macrocode}
% 
% And make the code to be in typewriter font:
%    \begin{macrocode}
\lstset{basicstyle=\ttfamily,columns=fullflexible}
%    \end{macrocode}
% 
%
%
%\subsection{Bibliography}
%\label{sec:biblio}
%
% \begin{macro}{\bibsection}
% \changes{v0.6}{2008/03/22}{Introduced the macro} 
%   We actually use chapter for bibliographic sections, but here we
%   try to be nice to \progname{natbib}:
%    \begin{macrocode}
\AtBeginDocument{
  \def\bibsection{%
    \clearpage
    {\pagestyle{empty}\cleardoublepage}%
    \everypar{}%
    \thispagestyle{empty}%
    \global\@topnum\z@
    \@afterindentfalse
    \gdef\@chapterart{}%
    \@nschapterpreamble
    \vspace*{0.622in}%
    \addcontentsline{toc}{chapter}{\MakeUppercase{\bibname}}%
    \addcontentsline{tbc}{bchapter}{\bibname}%
    {\centering
      {\fontfamily{\dgdefault}\fontseries{br}\fontsize{16pt}{20pt}\selectfont
        \MakeUppercase{\bibname}\par}}\nobreak\vskip25pt\relax%
    \@mkboth{\MakeUppercase\bibname}{\MakeUppercase\bibname}}}%
%    \end{macrocode}
%   
% \end{macro}
%
%
% \begin{macro}{thebibliography}
% \changes{v0.6}{2008/03/22}{Introduced the macro} 
%   We want bibliography to be in the table of contents:
%    \begin{macrocode}
\renewenvironment{thebibliography}[1]{%
  \bibsection
      \list{\@biblabel{\@arabic\c@enumiv}}%
           {\settowidth\labelwidth{\@biblabel{#1}}%
            \leftmargin\labelwidth
            \advance\leftmargin\labelsep
            \@openbib@code
            \usecounter{enumiv}%
            \let\p@enumiv\@empty
            \renewcommand\theenumiv{\@arabic\c@enumiv}}%
      \sloppy
      \clubpenalty4000
      \@clubpenalty \clubpenalty
      \widowpenalty4000%
      \sfcode`\.\@m}
     {\def\@noitemerr
       {\@latex@warning{Empty `thebibliography' environment}}%
      \endlist}
%    \end{macrocode}
%   
% \end{macro}
%
%\subsection{Index}
%\label{sec:index}
% \begin{macro}{theindex}
% \changes{v0.6}{2008/03/22}{Introduced the macro} 
%  We want our index to be in the table of contents too:
%    \begin{macrocode}
\renewenvironment{theindex}{%
  \clearpage
  {\pagestyle{empty}\cleardoublepage}%
  \everypar{}%
  \thispagestyle{empty}%
  \global\@topnum\z@
  \@afterindentfalse
  \gdef\@chapterart{}%
  \if@twocolumn
  \@restonecolfalse
  \else
  \@restonecoltrue
  \fi
  \twocolumn[%
  \@nschapterpreamble
  \vspace*{0.622in}%
  \addcontentsline{toc}{chapter}{\MakeUppercase{\indexname}}%
  \addcontentsline{tbc}{bchapter}{\indexname}%
  {\centering
    {\fontfamily{\dgdefault}\fontseries{br}%
      \fontsize{16pt}{20pt}\selectfont
      \MakeUppercase{\indexname}\par}}%
  \nobreak\vskip25pt\relax]%
  \@mkboth{\MakeUppercase\indexname}%
  {\MakeUppercase\indexname}%
  \parindent\z@
  \parskip\z@ \@plus .3\p@\relax
  \columnseprule \z@
  \columnsep 0.25in%
  \let\item\@idxitem}
{\if@restonecol\onecolumn\else\clearpage\fi}
%    \end{macrocode}
% \end{macro}
%
% \begin{macro}{\@idxitem}
% \changes{v0.6}{2008/03/22}{Changed default} 
%   This is how we format level 1 index entries:
%    \begin{macrocode}
\renewcommand\@idxitem{%
  \par\hangindent0.45in\normalfont\fontsize{9.5\p@}{11.5\p@}\selectfont
  \raggedright}
%    \end{macrocode}
% \end{macro}
% \begin{macro}{\subitem}
% \changes{v0.6}{2008/03/22}{Changed default} 
%   Level 2 entries:
%    \begin{macrocode}
\renewcommand\subitem{\@idxitem \hspace*{0.15in}}
%    \end{macrocode}   
% \end{macro}
% \begin{macro}{\subsubitem}
% \changes{v0.6}{2008/03/22}{Changed default} 
%   Level 3 entries:
%    \begin{macrocode}
\renewcommand\subsubitem{\@idxitem \hspace*{0.3in}}
%    \end{macrocode}   
% \end{macro}
% \begin{macro}{\indexspace}
% \changes{v0.6}{2008/03/22}{Changed default} 
%   The space between the groups
%    \begin{macrocode}
\renewcommand\indexspace{\par \vskip 14\p@\relax}
%    \end{macrocode}   
% \end{macro}
% \begin{macro}{\indexgroup}
% \changes{v0.6}{2008/03/22}{Introduced the macro} 
%   When an index starts a new letter, we call this macro:
%    \begin{macrocode}
\newcommand\indexgroup[1]{{\fontfamily{\sfdefault}%
    \fontseries{bc}\fontsize{11\p@}{13\p@}\selectfont#1}\par\vskip4\p@}
%    \end{macrocode}   
% \end{macro}
%
%
%
%\subsection{Update Page and Colophon}
%\label{sec:update_colophon}
%
% \begin{macro}{\updatesname}
% \changes{v0.6}{2008/03/22}{Introduced the macro} 
%   Just the name for updates:
%    \begin{macrocode}
\def\updatesname{Updates}
%    \end{macrocode}
%   
% \end{macro}
%
%
% \begin{macro}{\updatespage}
% \changes{v0.6}{2008/03/22}{Introduced the macro} 
%   This is the command that start update page.  It looks like
%   chapter, but it does not mark the pages or appear in the tables of
%   contents:
%    \begin{macrocode}
\newcommand\updatespage{%
  \clearpage
  {\pagestyle{empty}\cleardoublepage}%
  \everypar{}%
  \thispagestyle{empty}%
  \global\@topnum\z@
  \gdef\@chapterart{}%
  \vspace*{1.875in}%
  {\centering
    \fontfamily{\dgdefault}\fontseries{br}%
    \fontsize{16pt}{20pt}\selectfont
    \MakeUppercase{\updatesname}\par\nobreak\vskip40\p@}%
  \@nobreaktrue
  \everypar{%
    \if@nobreak
    \@nobreakfalse
    \clubpenalty \@M
    \setbox\z@\lastbox
    \else
    \clubpenalty \@clubpenalty
    \everypar{}%
    \fi}}
%    \end{macrocode}   
% \end{macro}
%
%
% \begin{macro}{\colophonname}
% \changes{v0.6}{2008/03/22}{Introduced the macro} 
%   Just the name for colophon:
%    \begin{macrocode}
\def\colophonname{Colophon}
%    \end{macrocode}%   
% \end{macro}
%
%
% \begin{macro}{\colophon}
%   Colophon does not start a separate page:
% \changes{v0.6}{2008/03/22}{Introduced the macro} 
%    \begin{macrocode}
\newcommand\colophon{%
  \everypar{}%
  \global\@topnum\z@
  \gdef\@chapterart{}%
  \vspace*{0.875in}%
  {\centering
    \fontfamily{\dgdefault}\fontseries{br}%
    \fontsize{16pt}{20pt}\selectfont
    \MakeUppercase{\colophonname}\par\nobreak\vskip40\p@}%
  \@nobreaktrue
  \everypar{%
    \if@nobreak
    \@nobreakfalse
    \clubpenalty \@M
    \setbox\z@\lastbox
    \else
    \clubpenalty \@clubpenalty
    \everypar{}%
    \fi}\par
    The fonts used in \emph{\@title} are New Baskerville, Futura, The
    Sans Mono Condensed and Dogma.  The book was typeset with
    \LaTeXe{} package 
    \texttt{nostarch} by Boris Veytsman
    \emph{(\csname ver@nostarch.cls\endcsname).}\par} 
%    \end{macrocode}   
% \end{macro}
%
%\subsection{End of Class}
%\label{end}
%
%
%    \begin{macrocode}
%</class>
%    \end{macrocode}
%
%\subsection{Dealing with \progname{hyperref}}
%\label{sec:hyperref}
%
% \changes{v0.6}{2008/03/22}{Introduced package \progname{nshyper}} 
% First, we load \progname{hyperref} with right options.  I do not
% know why bookmarks do not work, but this is important to switch off:
%    \begin{macrocode}
%<*nshyper>
\RequirePackage[breaklinks,colorlinks,linkcolor=black,
  citecolor=black,pagecolor=black,urlcolor=black,hyperindex,
  bookmarks=false]{hyperref}
%    \end{macrocode}
% 
% \progname{hyperref} expects the starred parts and chapters to have
% only one argument---but we have two!
%    \begin{macrocode}
\AtBeginDocument{%
  \def\@schapter[#1]#2{%
    \H@old@schapter[#1]{#2}%
  \@nschapterpreamble}%
  \def\@spart[#1]#2{%
    \H@old@spart[#1]{#2}%
    \Hy@GlobalStepCount\Hy@linkcounter
    \xdef\@currentHref{part*.\the\Hy@linkcounter}%
    \Hy@raisedlink{\hyper@anchorstart{\@currentHref}\hyper@anchorend}}}%
%    \end{macrocode}
% We change the preamble for bibliography and index:
%    \begin{macrocode}
\def\@nschapterpreamble{%
  \begingroup
  \let\@mkboth\@gobbletwo
  \Hy@GlobalStepCount\Hy@linkcounter
  \xdef\@currentHref{\Hy@chapapp*.\the\Hy@linkcounter}%
  \Hy@raisedlink{\hyper@anchorstart{\@currentHref}\hyper@anchorend}%
  \endgroup}
%    \end{macrocode}
% 
% We can add information about the file to the pdf metadata as well:
%    \begin{macrocode}
\let\ns@old@maketitle\maketitle
\def\maketitle{%
  \hypersetup{pdfauthor=\@author, pdftitle=\@title}%
  \ns@old@maketitle}
%</nshyper> 
%    \end{macrocode}
% 
%
%\subsection{\progname{MakeIndex} Style File}
%\label{sec:makeindex}
% \changes{v0.6}{2008/03/23}{Added \progname{MakeIndex} style} 
%
% These lines will produce some warnings when
% running Makeindex as they try to cover two different
% versions of the program:
%    \begin{macrocode}
%<*ist>
lethead_prefix   "\\indexgroup{"
lethead_suffix   "}\\nopagebreak\n"
lethead_flag       1
heading_prefix   "\\indexgroup{"
heading_suffix   "}\\nopagebreak\n"
headings_flag       1
%</ist>
%    \end{macrocode}
%
%
%\Finale
%\clearpage
%
%\PrintChanges
%\clearpage
%\PrintIndex
%
\endinput
